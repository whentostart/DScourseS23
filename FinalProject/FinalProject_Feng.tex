\documentclass[12pt,english]{article}
\usepackage{mathptmx}

\usepackage{color}
\usepackage[dvipsnames]{xcolor}
\definecolor{darkblue}{RGB}{0.,0.,139.}

\usepackage[top=1in, bottom=1in, left=1in, right=1in]{geometry}

\usepackage{amsmath}
\usepackage{amstext}
\usepackage{amssymb}
\usepackage{setspace}
\usepackage{lipsum}

\usepackage[authoryear]{natbib}
\usepackage{url}
\usepackage{booktabs}
\usepackage[flushleft]{threeparttable}
\usepackage{graphicx}
\usepackage[english]{babel}
\usepackage{pdflscape}
\usepackage[unicode=true,pdfusetitle,
 bookmarks=true,bookmarksnumbered=false,bookmarksopen=false,
 breaklinks=true,pdfborder={0 0 0},backref=false,
 colorlinks,citecolor=black,filecolor=black,
 linkcolor=black,urlcolor=black]
 {hyperref}
\usepackage[all]{hypcap} % Links point to top of image, builds on hyperref
\usepackage{breakurl}    % Allows urls to wrap, including hyperref

\linespread{2}

\begin{document}

\begin{singlespace}
\title{The Supplier and Customer Relationship and
Stock Return Comovement}
\end{singlespace}

\author{Qinghua Feng\thanks{Department of Accounting, University of Oklahoma.\
E-mail~address:~\href{mailto:student.name@ou.edu}{qhfeng@ou.edu}}}

% \date{\today}
\date{May 9, 2023}

\maketitle

\begin{abstract}
\begin{singlespace}
This study aims to investigate how the relationship between supplier and customer influences their stock return comovement. Using the data sets of FactSet and CRSP, we employ a multivariate regression framework to estimate the extent of stock return comovement between supplier and customer. Our findings suggest a positive relationship between supplier and customer stock returns, indicating that supply and customer relationships have a significant impact on a firm's financial performance. The results have important implications for firms in managing their supply chain.
\end{singlespace}

\end{abstract}
\vfill{}


\pagebreak{}


\section{Introduction}\label{sec:intro}
As a crucial part of the supply chain, the relationship between supplier and customer has become increasingly important across countries. The collaboration between a supplier and a customer has a significant impact on the shipment, the cost, and the quality of goods, which in turn determines each other's financial performance. As such, I predict the stock return comovement between supplier and customer has a positive relationship, as stock returns are closely linked to a firm's financial performance.

From the business world to academic research, there has been a growing interest in understanding the impact of supply chain relationships on a firm's stock returns. While prior research has primarily focused on the accounting comparability of supplier and customer and the stock return comovement of firms followed by common analysts or investors.~\citep{pindyck1993comovement} Pindyck and Rotemberg (1993) find that firms predominantly invested by institutional investors exhibit greater comovement. While there is rare research that has been dived into the relationship between supplier and customer stock returns. This gap motivates me to investigate the stock returns comovement between supplier and customer.

Specifically, this study focuses on the supply chain relationship between a supplier and a customer, as this relationship is a critical component of the supply chain. The supplier-customer relationship is characterized by a high level of interdependence, with each party relying on the other to fulfill their respective needs. As such, it is important to understand how changes in this relationship can affect the financial performance of both parties.

In this study, I aim to fill this gap in the literature by examining how the relationship between supplier and customer influences their stock return comovement. I use a multivariate regression framework to estimate the extent of stock return comovement between supplier and customer using the data sets of FactSet and CRSP. I find a positive and significant relationship between supplier and customer stock returns, indicating that supply and customer relationships have a significant impact on a firm's financial performance. 

This study adds to the literature on the supply chain relationship and stock return comovement. The results of our study have important implications for firms in managing their supply chain and for investors in assessing the performance of their investment portfolios.


\section{Literature Review}\label{sec:litreview}

Stock return is a comprehensive reflection of economic events. The determinants of stock returns comovement have been demonstrated as common risk factors, shared investors, and disclosure similarity.

Return comovement reflects fundamental economic events and the underlying operational performance. According to \cite{fama1993common} (Fama French 1993), firms' stock prices are more likely to move together if they are exposed to common risk factors. Comovement tends to increase in challenging times when firms face similar economic pressures ~\citep {erb1994forecasting} (Erb et al. 1994, Ding et al. 2011).

In addition to the research on common risk factors, other determinants of return comovement have been explored. Studies have found that firms predominantly invested by institutional investors exhibit greater comovement ~\citep{pindyck1993comovement} (Pindyck and Rotemberg 1993). Furthermore, shared ownership ~\citep{anton2014connected} (Antón and Polk 2014), investor attention ~\citep{drake2017comovement} (Drake et al. 2017), and analyst following ~\citep{muslu2014sell} (Muslu et al. 2014) have also been identified as determinants of stock return comovement.

Researchers have also examined the correlation between stock returns and firms' disclosures. ~\citep{dyer2017evolution} (Dyer et al. 2017) expect that firms' financial reporting choices influence the comovement in stock returns and find that the similarity of firms' disclosures not only predicts but also influences future return comovement. Other factors, such as information intermediaries ~\citep{piotroski2004influence} (Piotroski 2004), analysts' ratings of firms' disclosure policies ~\citep{haggard2008does} (Haggard et al. 2008), and commonality in news coverage \citep{dang2015commonality} (Dang et al. 2015), have also been studied.

In addition to the determinants of stock return comovement identified in prior studies, there is growing interest in understanding the impact of supply chain relationship on stock returns. A firm's relationships with its suppliers and customers are critical components of the supply chain and can have a significant impact on its financial performance. ~\citep{lavassani2021firm} Lavassani and Ovahedi (2021) provide insights into the relationship between a firm's location characteristics in the global supply chain ecosystem and its asset performance.  Previous research finds that supply chain structure is closely related to firm returns, with a direct impact from supplier and customer returns and a systemic impact from exposures through the network (Wu, Birge 2014) ~\citep{wu2014supply}.

The collaborations and trades between suppliers and customers could influence each other. If customers have more demands, suppliers could sell more. Instead, suppliers' revenue might stagger without customers' demand growth. Furthermore, the performance effects will become profound as the relationship gets longer. Therefore, the hypothesis is predicted as below:


\textbf{H1: longer supplier and customer relationships will have a stronger impact on stock return 
    comovement}
    

\section{Data}\label{sec:data}
The main databases used in our empirical analysis are the CRSP stock database
(for stock returns), and the FactSet Revere relationships database (for information about
suppliers, customers, and competitors).
The FactSet Revere database provides arguably the most comprehensive
coverage of firm-level supplier-customer relationships that is currently
available. It includes relationships disclosed by either suppliers or customers
(or by both), with the start and end dates for each relationship. FactSet’s analysts
monitor the relationships data on a regular basis. The comprehensive supplier-customer data allows me to measure the relationship length between supplier and customer. 

The sample period is from April 2003, when the FactSet database started, to
April 2023. If the relationship ending dates are missing, I replace them with the recent date of April 25, 2023. I use the start date and end date to calculate the supplier and customer relationship length. I filter the observations that the supplier and customer relationship length is greater than 24 months to make sure there is no overlap between the treatment and control groups. 

FactSet Revere database was merged with CRSP databases using FactSet-CRSP linking table. I use the start dates in FactSet to find their following 12 months' stock returns CRSP (control group), and the end dates in FactSet to find their previous 12 months' stock returns CRSP (treatment group). After stock returns are matched, their correlation was calculated. Then the market value of suppliers and customers was merged. The final total observations are 25,391.

I run two regressions, without and with control variables of market value. The results are both positive and significant.

This study provides a correlation between supplier-customer relationships and stock returns, but it is challenging to establish causality. There may be other factors that could impact stock returns that are not accounted for in the analysis.

\section{Empirical Methods}\label{sec:methods}
Since supply chain management is an essential part of a firm's operation, the impact of trading partners on each other's financial performance is supposed to be profound. To identify such impact, I construct each unique pair of supplier and customer as an observation. The observations 12 months after starting the relationship as post=0 and the observations 12 months before the ending date of the relationship as post=1. 
The primary empirical model aims to examine the relationship between supplier-customer relationships and stock returns, which can be depicted in the following equation:

\begin{equation}
\label{eq:1}
Y_{i,t}=\alpha_{0} + \beta_{1}Post_{0,1} + \varepsilon_{i,t},
\end{equation}
\begin{equation}
\label{eq:2}
Y_{i,t}=\alpha_{0} + \beta_{1}Post_{0,1} +\beta_{2}SMV_{i,t}+\beta_{3}CMV_{i,t}+ \varepsilon_{i,t},
\end{equation}

where $Y_{i,t}$ is the correlation between the stock returns of each pair of supplier and customer during their relationship period $t$. The model includes four main variables:

$\alpha_{0}$ represents the intercept, which is the expected value of $Y$ when all other variables are zero.

$Post_{0,1}$ is a dummy variable that takes a value of one if the observation is during the 12 months before the end of the relationship ($post=1$) and zero if the observation is during the 12 months after the start of the relationship ($post=0$). This variable captures the difference in stock returns between the short and long supplier-customer relationships.

$SMV_{i,t}$ represents the log of the market value of the supplier in their relationship period $t$. This variable is included as a control variable to account for differences in the size and growth prospects of suppliers.

$CMV_{i,t}$ represents the log of the market value of the customer in their relationship period $t$. This variable is also included as a control variable to account for differences in the size and growth prospects of customers.

$\varepsilon_{i,t}$ represents the error term, which captures the unobserved factors that influence stock returns.

The inclusion of $Post_{0,1}$ allows the study to compare the stock returns of supplier-customer pairs of shorter and longer relationships. Market value is often used as a control variable in financial studies because it reflects a firm's size and growth prospects, which can impact its stock returns. In this study, market value is used as a control variable to account for differences in the size and growth prospects of the suppliers and customers. By controlling for $SMV_{i,t}$ and $CMV_{i,t}$, the study can more accurately isolate the impact of supplier-customer relationships on stock returns.

Overall, the model aims to investigate the impact of supplier-customer relationships on stock returns while controlling for other factors that could influence the results.


\section{Research Findings}\label{sec:results}
The main results are reported in Tables from table 1 to table 4.

Table 1 presents the results of the Analysis of Variance (ANOVA) for MODEL1. The F value for Model1 is 196.93 with a very low p-value ($\textless$ .0001), indicating that there is a significant difference between groups and that Model1 provides a good fit for the data. The Error term represents unexplained variability in the data and has an associated mean square (MS) of 0.11432.

Table 2 presents the parameter estimates for MODEL1. The Intercept parameter represents the expected value of the response variable when all predictor variables are equal to zero. In this case, the Intercept has an estimated value of 0.24859 with a standard error of 0.00291 and a t-value of 85.29, indicating that it is statistically significant ($\textless$ .0001).

The post parameter represents the effect of a binary predictor variable on the response variable. In this case, post1 has an estimated value of 0.06079 with a standard error of 0.00433 and a t-value of 14.03, indicating that it is also statistically significant ($\textless$ .0001). This suggests that there is a positive relationship between post1 and the response variable.

Overall, Table 2 provides important information about the estimated values and statistical significance of the parameters in MODEL1. These estimates can be used to make predictions about future observations or to test hypotheses about relationships between variables in the model.

Table 3 presents the results of the Analysis of Variance (ANOVA) for MODEL2. The F value for Model2 is 399.87 with a very low p-value ($\textless$ .0001), indicating that there is a significant difference between groups and that Model2 provides a good fit for the data. The Error term represents unexplained variability in the data and has an associated mean square (MS) of 0.10988. The R-Square value for MODEL2 is 0.0465, which indicates that only 4.65% of the total variation in the response variable can be explained by this model.

Overall, Table 3 provides important information about the statistical significance of Model2 in explaining the variability in the data being analyzed. It also shows that there is still a significant amount of unexplained variability in the data, suggesting that other factors may be influencing the response variable as well.

Table 4 presents the parameter estimates for MODEL2. The post parameter represents the effect of a binary predictor variable on the response variable. In this case, post has an estimated value of 0.05513 with a standard error of 0.00425 and a t-value of 12.97, indicating that it is also statistically significant ($\textless$ .0001). This suggests that there is a positive relationship between post1 and the response variable.

The CMV parameter represents the effect of a continuous predictor variable on the response variable. In this case, CMV has an estimated value of 0.02785 with a standard error of 0.00090264 and a t-value of 30.85, indicating that it is also statistically significant ($\textless$ .0001). This suggests that there is a positive relationship between meanc mv1 and the response variable.

The SMV parameter also represents the effect of a continuous predictor variable on the response variable. In this case, SMV has an estimated value of 0.00358 with a standard error of 0.00081374 and a t-value of 4.40, indicating that it is also statistically significant ($\textless$ .0001). This suggests that there is also a positive relationship between means mv1 and the response variable.

Overall, Table 4 provides important information about the estimated values and statistical significance of the parameters in MODEL2. These estimates can be used to make predictions about future observations or to test hypotheses about relationships between variables in the model.


\section{Conclusion}\label{sec:conclusion}


This study investigates the relationship between supplier and customer and its impact on comovement. The findings suggest a positive relationship between the length of supplier-customer relationships and their stock returns, indicating that the supply and customer relationships significantly impact a firm's financial performance. The study fills a gap in the literature by investigating the length of relationship and stock return comovement between supplier and customer, which has been underexplored compared to other factors. The research has important implications for firms supply chain management and for investors in assessing their investment portfolios.

The study uses a multivariate regression framework to estimate the extent of stock return comovement between supplier and customer using datasets from FactSet and CRSP. The ANOVA table indicates that the regression model is statistically significant, and the coefficient of determination is low since only dummy variables are included. The parameter estimates show that the intercept is statistically significant, and the coefficient for "post" is also statistically significant, indicating a positive relationship between the relationship and stock return comovement.

Future research can consider incorporating other control variables, such as market value, to enhance the model's predictive power. Additionally, other factors, such as information intermediaries and commonality in news coverage, could be explored to provide a more comprehensive understanding of the determinants of stock return comovement.










\vfill
\pagebreak{}
\begin{spacing}{1.0}
\bibliographystyle{plain}
\bibliography{PS11_Feng}
\addcontentsline{toc}{section}{References}
\end{spacing}

\vfill
\pagebreak{}
\clearpage

%========================================
% FIGURES AND TABLES 
%========================================
\section*{Figures and Tables}\label{sec:figTables}
\addcontentsline{toc}{section}{Figures and Tables}
%----------------------------------------
% Figure 1
%----------------------------------------
\begin{figure}[ht]
\centering
\bigskip{}

\includegraphics[width=.9\linewidth]{model1.png}
\caption{Residuals for Returns}
\label{fig:figure1}


\includegraphics[width=.9\linewidth]{model2.png}
\caption{Residual by Regressors for Returns}
\label{fig:figure2}


\end{figure}




\begin{table}[h]
\centering
\caption{Analysis of Variance for MODEL1}
\label{tab:anova}
\begin{tabular}{lrrrrr}
\hline
Source & DF & Sum of Squares & Mean Square & F Value & Pr $>$ F \\
\hline
Model & 1 & 22.51273 & 22.51273 & 196.93 & $<$.0001 \\
Error & 24586 & 2810.63329 & 0.11432 \\
Corrected Total & 24587 & 2833.14602 \\
\hline
Root MSE & \multicolumn{5}{r}{0.33811} \\
R-Square & \multicolumn{5}{r}{0.0079} \\
Dependent Mean & \multicolumn{5}{r}{0.27611} \\
Adj R-Sq & \multicolumn{5}{r}{0.0079} \\
Coeff Var (\%)& \multicolumn{5}{r}{122.45478} \\
\hline
\end{tabular}
\end{table}



\begin{table}[h]
\centering
\caption{Parameter Estimates for MODEL1}
\label{tab:estimates}
\begin{tabular}{|c|c|c|c|c|c|}
\hline
Variable & DF & Parameter Estimate & Standard Error & t Value & Pr $>$ $|$t$|$ \\
\hline
Intercept & 1 & 0.24859 & 0.00291 & 85.29 & $<$.0001 \\
post & 1 & 0.06079 & 0.00433 & 14.03 & $<$.0001 \\
\hline
\end{tabular}
\end{table}




\begin{table}[h]
\centering
\caption{Analysis of Variance for MODEL2}
\label{tab:anova}
\begin{tabular}{lrrrrr}
\hline
Source & DF & Sum of Squares & Mean Square & F Value & Pr $>$ F \\
\hline
Model & 3 & 131.81361 & 43.93787 & 399.87 & $<$.0001 \\
Error & 24584 & 2701.33242 & 0.10988 \\
Corrected Total & 24587 & 2833.14602 \\
\hline
Root MSE & \multicolumn{5}{r}{0.33148} \\
R-Square & \multicolumn{5}{r}{0.0465} \\
Dependent Mean & \multicolumn{5}{r}{0.27611} \\
Adj R-Sq & \multicolumn{5}{r}{0.0464} \\
Coeff Var (\%)& \multicolumn{5}{r}{120.05502} \\
\hline
\end{tabular}
\end{table}


\begin{table}[h]
\centering
\caption{Parameter Estimates for MODEL2}
\label{tab:estimates}
\begin{tabular}{|c|c|c|c||c||c|}
\hline
Variable & DF & Parameter Estimate & Standard Error & t Value & Pr $>$ $|$t$|$\\
\hline
Intercept & 1 & -0.18589 & 0.01748 & -10.64 & $<$.0001 \\
post & 1 & 0.05513 & 0.00425 & 12.97 & $<$.0001 \\
CMV & 1 & 0.02785 & 0.00090264 & 30.85 & $<$.0001 \\
SMV & 1 & 0.00358& 0.00081374 & 4.40 & $<$.0001 \\
\hline
\end{tabular}
\end{table}


\end{document}
